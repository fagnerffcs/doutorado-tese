
% ----------------------------------------------------------
% DADOS DO TRABALHO - CAPA e FOLHA DE ROSTO
% Configure os dados do trabalho aqui
% ----------------------------------------------------------
\titulo{\textbf{PANOPTES:} An Open-Source Reference Architecture for Observability in Kubernetes Ecosystems}
\autor{Fagner Fernandes Candido da Silva}
\local{Recife}
\data{\Year}
\areaconcentracao{\textbf{Concentration Area}: Software Engineering and Programming Languages}
\orientador{\textbf{Advisor}: Vinícius Cardoso Garcia}
\coorientador{\textbf{Coadvisor}: Jackson Raniel Florêncio da Silva}

\instituicao{UNIVERSIDADE FEDERAL DE PERNAMBUCO \\ CENTRO DE INFORMÁTICA \\PROGRAMA DE PÓS-GRADUAÇÃO EM CIENCIAS DA COMPUTAÇÃO}
\departamento{Centro de Informática}
\programa{Pós-graduação em Ciencias da Computacao}
\emailprograma{email@cin.ufpe.br}
\siteprograma{http://cin.ufpe.br/\textasciitilde posgraduacao}

\tipotrabalho{Tese de Doutorado}
% O preambulo deve conter o tipo do trabalho, o objetivo, 
% o nome da instituição e a área de concentração 
%\preambulo{Trabalho apresentado ao Programa de Pós-graduação em Ciência da Computação do Centro de Informática da Universidade Federal de Pernambuco, como requisito parcial para obtenção do grau de Mestre Profissional em Ciência da Computação.}

%\preambuloatadefesa{Dissertação apresentada ao Programa de Pós-Graduação Profissional em Ciência da Computação da Universidade Federal de Pernambuco, como requisito parcial para a obtenção do título de Mestre Profissional em 04 de setembro de 2020.}

\preambulo{Doctoral thesis presented to the Pos Graduate Program in Computer Science at the Federal University of Pernambuco, as a partial requirement for obtaining the title of Doctor of Computer Science.}

\preambuloatadefesa{Doctoral thesis presented to the Pos Graduate Program in Computer Science at the Federal University of Pernambuco, as a partial requirement for obtaining the title of Doctor of Computer Science.}




\input{userlists}



