\section{Design Science Research}
\label{sec:dsr-approach}

Unlike traditional natural sciences that seek to understand reality (positivism) or social sciences that seek to interpret it (interpretivism), Design Science is rooted in the pragmatic epistemology. Its fundamental goal is to change reality through the creation of innovative artifacts.

As defined by \citeonline{simon_1969} and consolidated by \citeonline{hevner_2004}, DSR involves the rigorous process of designing artifacts to solve observed problems, make research contributions, and evaluate the designs. In the context of this thesis, the primary artifact is PANOPTES, a reference architecture for observability.

According to the classification by \citeonline{vaishnavi_2008}, the desired outputs of DSR are classified by level of abstraction as described in table \ref{tab:dsr-outputs}.

\begin{table}[ht]
\centering
\caption{The Outputs of Design Science Research}
\label{tab:dsr-outputs}
\begin{tabularx}{\textwidth}{|l|X|}
\hline
\textbf{Output} & \textbf{Description} \\ \hline
Constructs & The conceptual vocabulary of a domain \\ \hline
Models & A set of propositions or statements expressing relationships between constructs \\ \hline
Methods & A set of steps used to perform a task -- how-to knowledge \\ \hline
Instantiations & The operationalization of constructs, models, and methods \\ \hline
Better theories & Artifact construction as analogous to experimental natural science \\ \hline
\end{tabularx}
\par\medskip\ABNTEXfontereduzida\selectfont\textbf{Source:} Adapted from \citeonline{vaishnavi_2008} \par\medskip
\end{table}

The artifacts produced in this research are of two types:
\begin{itemize}
    \item \textbf{Architecture:} A high-level structural framework defining components (Metrics, Logs, Traces) and their relationships.
    \item \textbf{Instantiation:} A concrete implementation (Infrastructure as Code) that operationalizes the architecture in a working environment.
\end{itemize}

\subsection{Research Cycles}
\label{subsec:research-cycles}

Following the framework proposed by \citeonline{hevner_2004}, this research operates across three cycles as depicted in figure \ref{fig:dsr-cycles} :

\begin{figure}[ht]
    \centering
    \caption{Design science research cycles}
    \label{fig:dsr-cycles}
    \includegraphics[width=1.0\linewidth]{images/methodology/hevner-dsr-cycles.png}
    \par\medskip\ABNTEXfontereduzida\selectfont\textbf{Source:} \cite{hevner_2004} \par\medskip
\end{figure}

\subsection{Research Cycles Instantiation}
\label{subsec:cycles-instantiation}

To operationalize the DSR framework in the context of this thesis, we instantiated the research cycles focusing on the specific problem of observability in Kubernetes. Figure \ref{fig:panoptes-cycles} illustrates how the Knowledge Base feeds the Design Cycle (Rigor) and how the results contribute back to the scientific body of knowledge.

\begin{figure}[ht]
    \centering
    \caption{Instantiation of DSR Cycles for PANOPTES Research}
    \label{fig:panoptes-cycles}
    \includegraphics[width=1.0\textwidth]{images/methodology/design-cycle.png}
    \par\medskip
    \small\textbf{Source:} Author (2026)
\end{figure}

\begin{enumerate}
    \item \textbf{Relevance Cycle:} Connects the research to the environment (Cloud-Native context). We identified the problem of ``observability fragmentation'' and defined requirements based on the needs of practitioners (Chapter \ref{chap:observability} and \ref{chap:cloud-computing}).
    \item \textbf{Design Cycle:} The core iterative process of building the PANOPTES architecture and its Ansible/Helm instantiation (Chapter \ref{chap:experiment-implementation}).
    \item \textbf{Rigor Cycle:} Connects the research to the knowledge base. Using Kalman's Control Theory and Distributed Systems theory to ground our architectural decisions (Chapter \ref{chap:observability} and \ref{chap:cloud-computing}).
\end{enumerate}

\subsection{Synthesis of Design Knowledge}
\label{subsec:design-knowledge}

One of the primary goals of DSR is to generate knowledge that extends beyond the specific instantiation of the artifact. According to \citeonline{gregor_2013}, this knowledge is crystallized in the form of Design Principles and Contingency Heuristics.

In this thesis, these elements are defined as follows:

\begin{itemize}
    \item \textbf{Design Principles:} Fundamental rules that guide the construction of the artifact to ensure it addresses the research problem. These are derived from the Knowledge Base (e.g., Control Theory, Distributed Systems Patterns) during the Rigor Cycle.
    \item \textbf{Contingency Heuristics:} Context-aware rules of thumb (e.g., ``If condition $X$ occurs in the cluster, apply strategy $Y$'') that guide practitioners in adapting the architecture to specific scenarios.
\end{itemize}

The validity of these principles is established through the Design Cycle (construction) and verified by the Evaluation Step (Controlled Experiment). The formalized principles and heuristics are presented as the final contribution of the Knowledge Cycle in Chapter \ref{chap:conclusion}.