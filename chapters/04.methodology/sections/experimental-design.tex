\section{Experimental Design}
\label{sec:experimental-design}

To test these hypotheses, we adopt a 2x2 Factorial Design, crossing two factors: Architecture Type (Vanilla vs. PANOPTES) and System State (Idle vs. Chaos). This results in four distinct experimental scenarios, as detailed below:

\subsection{Scenario 1: Baseline Idle (Control Group)}
\begin{itemize}
    \item \textbf{Definition:} A standard ``Vanilla'' Kubernetes installation (created via Vagrant/Ansible) running the target microservices application without any observability agents (Prometheus, Jaeger, Loki are absent).
    \item \textbf{Objective:} To establish the baseline resource consumption of the Kubernetes orchestrator and the application under normal conditions. This serves as the reference point for calculating overhead.
\end{itemize}

\subsection{Scenario 2: PANOPTES Idle (Experimental Group)}
\begin{itemize}
    \item \textbf{Definition:} The same Kubernetes cluster, but with the full PANOPTES architecture instantiated (OpenTelemetry Collectors, Prometheus, Loki, Tempo, Grafana). No faults are injected.
    \item \textbf{Objective:} To measure the "static cost" of observability. By comparing Scenario 2 with Scenario 1, we isolate the specific CPU and Memory overhead generated by the telemetry collection stack.
\end{itemize}

\subsection{Scenario 3: Baseline Under Stress (Control Group)}
\begin{itemize}
    \item \textbf{Definition:} The Vanilla Kubernetes environment subjected to \textit{Chaos Engineering} experiments (e.g., Pod Kill, Network Latency, CPU Stress) injected via Chaos Mesh.
    \item \textbf{Objective:} To evaluate the diagnostic difficulty using only native tools (e.g., \texttt{kubectl}, standard logs). In this scenario, we measure the time required to identify the root cause of the failure using only CLI commands and unstructured logs.
\end{itemize}

\subsection{Scenario 4: PANOPTES Under Stress (Experimental Group)}
\begin{itemize}
    \item \textbf{Definition:} The PANOPTES environment subjected to the exact same \textit{Chaos Engineering} experiments as Scenario 3.
    \item \textbf{Objective:} To evaluate the diagnostic efficiency using the proposed architecture. We measure the time required to identify the root cause using the unified dashboards (Grafana), traces (Tempo), and correlated metrics (Prometheus).
\end{itemize}