\section{Research Hypotheses}
\label{sec:hypotheses}

Based on the Research Questions defined in Chapter \ref{chap:intro}, we formulate two primary hypotheses to be tested. The null hypothesis ($H_0$) represents the status quo (Vanilla Kubernetes), while the alternative hypothesis ($H_1$) represents the expected improvement offered by the PANOPTES architecture.

\subsection{Hypothesis 1: Diagnostic Efficiency}
We postulate that a unified observability architecture significantly reduces the time required to detect and diagnose system anomalies compared to traditional CLI-based monitoring.

\begin{itemize}
    \item \textbf{Null Hypothesis ($H_{0a}$):} There is no statistically significant difference in the Mean Time to Detect (MTTD) faults between the Vanilla Kubernetes environment and the PANOPTES environment.
    \item \textbf{Alternative Hypothesis ($H_{1a}$):} The PANOPTES environment demonstrates a statistically significant reduction in MTTD compared to the Vanilla Kubernetes environment.
\end{itemize}

\subsection{Hypothesis 2: Resource Overhead}
We postulate that the resource consumption (CPU and Memory) introduced by the observability agents is negligible compared to the operational benefits provided.

\begin{itemize}
    \item \textbf{Null Hypothesis ($H_{0b}$):} The resource overhead (CPU/Memory) introduced by the PANOPTES architecture exceeds 15\% of the total cluster capacity, degrading application performance.
    \item \textbf{Alternative Hypothesis ($H_{1b}$):} The resource overhead introduced by the PANOPTES architecture remains below 15\% of the total cluster capacity, maintaining application performance stability.
\end{itemize}