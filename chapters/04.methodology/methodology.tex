\chapter{METHODOLOGY}
\label{chap:methodology}

This chapter outlines the methodological framework adopted to conduct this research. Given the nature of this study—which aims not only to explain a phenomenon but to create a solution (artifact) for a practical problem in Kubernetes environments—we adopt the Design Science Research (DSR) approach. DSR is an epistemological-methodological approach that allows for rigorous scientific research linked to the development of computational artifacts \cite{pimentel_2020}. It is especially suitable for complex and ill-defined problems, also defined as wicked problems \cite{marshall_2008}, that require innovative solutions \cite{hevner_2004}.

To demonstrate the suitability of this research method to this work, in the following sections, we will explain the details of how the DSR is characterized as a methodology applied to the field of information systems and the DSRM as an applicable research method.  

\section{Design Science Research}
\label{sec:dsr-approach}

Unlike traditional natural sciences that seek to understand reality (positivism) or social sciences that seek to interpret it (interpretivism), Design Science is rooted in the pragmatic epistemology. Its fundamental goal is to change reality through the creation of innovative artifacts.

As defined by \citeonline{simon_1969} and consolidated by \citeonline{hevner_2004}, DSR involves the rigorous process of designing artifacts to solve observed problems, make research contributions, and evaluate the designs. In the context of this thesis, the primary artifact is PANOPTES, a reference architecture for observability.

According to the classification by \citeonline{vaishnavi_2008}, the desired outputs of DSR are classified by level of abstraction as described in table \ref{tab:dsr-outputs}.

\begin{table}[ht]
\centering
\caption{The Outputs of Design Science Research}
\label{tab:dsr-outputs}
\begin{tabularx}{\textwidth}{|l|X|}
\hline
\textbf{Output} & \textbf{Description} \\ \hline
Constructs & The conceptual vocabulary of a domain \\ \hline
Models & A set of propositions or statements expressing relationships between constructs \\ \hline
Methods & A set of steps used to perform a task -- how-to knowledge \\ \hline
Instantiations & The operationalization of constructs, models, and methods \\ \hline
Better theories & Artifact construction as analogous to experimental natural science \\ \hline
\end{tabularx}
\par\medskip\ABNTEXfontereduzida\selectfont\textbf{Source:} Adapted from \citeonline{vaishnavi_2008} \par\medskip
\end{table}

The artifacts produced in this research are of two types:
\begin{itemize}
    \item \textbf{Architecture:} A high-level structural framework defining components (Metrics, Logs, Traces) and their relationships.
    \item \textbf{Instantiation:} A concrete implementation (Infrastructure as Code) that operationalizes the architecture in a working environment.
\end{itemize}

\subsection{Research Cycles}
Following the framework proposed by \citeonline{hevner_2004}, this research operates across three cycles as depicted in figure \ref{fig:dsr-cycles} :

\begin{figure}[ht]
    \centering
    \caption{Design science research cycles}
    \label{fig:dsr-cycles}
    \includegraphics[width=1.0\linewidth]{images/methodology/hevner-dsr-cycles.png}
    \par\medskip\ABNTEXfontereduzida\selectfont\textbf{Source:} \cite{hevner_2004} \par\medskip
\end{figure}

\begin{enumerate}
    \item \textbf{Relevance Cycle:} Connects the research to the environment (Cloud-Native context). We identified the problem of ``observability fragmentation'' and defined requirements based on the needs of practitioners (Chapter \ref{chap:observability} and \ref{chap:cloud-computing}).
    \item \textbf{Design Cycle:} The core iterative process of building the PANOPTES architecture and its Ansible/Helm instantiation (Chapter \ref{chap:experiment-implementation}).
    \item \textbf{Rigor Cycle:} Connects the research to the knowledge base. Using Kalman's Control Theory and Distributed Systems theory to ground our architectural decisions (Chapter \ref{chap:observability} and \ref{chap:cloud-computing}).
\end{enumerate}

\section{Design Science Research Methodology (DSRM)}
\label{sec:dsrm-process}

To ensure methodological rigor, we follow the Design Science Research Methodology (DSRM) process model proposed by \citeonline{peffers_2006}, which is structured in six sequential steps. As described by \cite{pimentel_2020}, DSRM method has been consolidated as a popular research method since 2018. Table \ref{tab:dsrm-mapping} maps these steps to the structure of this thesis.

\begin{table}[h]
\centering
\small
\caption{Mapping of DSRM Steps to Thesis Structure}
\label{tab:dsrm-mapping}
\begin{tabularx}{\textwidth}{|l|X|l|}
\hline
\textbf{DSRM Step} & \textbf{Description} & \textbf{Thesis Location} \\ \hline
1. Problem Identification & Definition of observability complexity and lack of blueprints. & Chapter \ref{chap:intro} \& \ref{chap:cloud-computing} \\ \hline
2. Define Objectives & Definition of requirements for a unified architecture. & Chapter \ref{chap:intro}, Section \ref{sec:problem} \\ \hline
3. Design \& Development & Creation of PANOPTES Architecture and IaC artifacts. & Chapter 5 \\ \hline
4. Demonstration & Deployment of the artifact in a controlled lab (Vagrant). & Chapter 5 \\ \hline
5. Evaluation & Controlled experiment using Chaos Engineering. & Chapter \ref{chap:methodology} Section \ref{sec:evaluation-strategy} \\ \hline
6. Communication & Thesis defense and academic publications. & All Chapters \\ \hline
\end{tabularx}
\par\medskip
\ABNTEXfontereduzida\selectfont\textbf{Source:} Author (2026) 
\end{table}

Another essential element of this research, it is the DSR Research Elements Mapping \cite{pimentel_2020}. This mapping distinguishes the artifact construction from the theoretical knowledge base and the application context, as detailed in Table \ref{tab:dsr-elements}.

\begin{table}[ht]
\centering
\caption{DSR Research Elements Mapping}
\label{tab:dsr-elements}
\begin{tabularx}{\textwidth}{|l|X|}
\hline
\textbf{DSR Element} & \textbf{Application in this Thesis} \\ \hline
\textbf{Context} & Cloud-native environments using Kubernetes; DevOps and SRE practitioners aiming for reliability. \\ \hline
\textbf{Problem} & Lack of unified reference architectures for observability leads to high Mean Time to Detect (MTTD) and operational complexity (Wicked Problem). \\ \hline
\textbf{Theoretical Conjectures} & Applying Control Theory concepts (observability pipelines) and centralized correlation reduces cognitive load and improves diagnostic efficiency ($H_{1a}$). \\ \hline
\textbf{Artifact} & \textbf{PANOPTES:} A reference architecture and its instantiation using IaC (Ansible/Helm) integrating OpenTelemetry, Prometheus, Loki, and Tempo. \\ \hline
\textbf{Evaluation Strategy} & Quantitative Controlled Experiment (2x2 Factorial Design) to validate Utility (MTTD) and Viability (Resource Overhead). \\ \hline
\end{tabularx}
\par\medskip\ABNTEXfontereduzida\selectfont\textbf{Source:} Adapted from \citeonline{pimentel_2020} \par\medskip
\end{table}

\section{Evaluation Strategy: The Controlled Experiment}
\label{sec:evaluation-strategy}

The fifth step of the DSRM (Evaluation) is critical to demonstrate the utility and validity of the artifact. To achieve this, we employ a quantitative Controlled Experiment strategy, comparing the proposed artifact (PANOPTES) against a baseline (Vanilla Kubernetes).

The evaluation is designed to verify if the artifact satisfies the research objectives (SOs), specifically regarding fault diagnosis efficiency and resource overhead.

\subsection{Research Hypotheses}
We formulate two primary hypotheses to test the artifact's effectiveness:

\begin{itemize}
    \item \textbf{Hypothesis 1 (Diagnostic Efficiency):}
    \begin{itemize}
        \item $H_{0a}$: There is no significant difference in Mean Time to Detect (MTTD) between Vanilla and PANOPTES.
        \item $H_{1a}$: PANOPTES significantly reduces MTTD compared to Vanilla Kubernetes.
    \end{itemize}
    
    \item \textbf{Hypothesis 2 (Resource Overhead):}
    \begin{itemize}
        \item $H_{0b}$: The artifact introduces an overhead exceeding 15\% of cluster capacity.
        \item $H_{1b}$: The artifact maintains resource overhead below 15\%, ensuring operational viability.
    \end{itemize}
\end{itemize}

\subsection{Experimental Design}
We adopt a **2x2 Factorial Design**, crossing two independent variables: \textbf{Architecture Type} (Factor A) and \textbf{System State} (Factor B).

\begin{table}[h]
\centering
\caption{Experimental Scenarios (2x2 Factorial Design)}
\label{tab:scenarios}
\begin{tabular}{|l|c|c|}
\hline
\textbf{State / Arch} & \textbf{Vanilla (Baseline)} & \textbf{PANOPTES (Artifact)} \\ \hline
\textbf{Idle} & Scenario 1 (Control) & Scenario 2 (Overhead Test) \\ \hline
\textbf{Chaos (Stress)} & Scenario 3 (Control) & Scenario 4 (Efficiency Test) \\ \hline
\end{tabular}
\end{table}

\subsubsection{Scenarios Definition}
\begin{itemize}
    \item \textbf{Scenario 1 (Baseline/Idle):} Standard Kubernetes running the workload without observability agents. Serves as the reference for resource usage.
    \item \textbf{Scenario 2 (Artifact/Idle):} Kubernetes with PANOPTES instantiated. Comparing S2 vs S1 isolates the "static cost" (overhead) of the artifact.
    \item \textbf{Scenario 3 (Baseline/Chaos):} Vanilla Kubernetes subjected to fault injection (Chaos Mesh). Diagnosis is attempted using only CLI tools (\texttt{kubectl}).
    \item \textbf{Scenario 4 (Artifact/Chaos):} PANOPTES environment subjected to the same faults. Diagnosis is performed using the unified dashboards and traces provided by the artifact.
\end{itemize}

\subsection{Metrics and Measurement}
To quantify the results, we utilize the metrics defined in Table \ref{tab:metrics}.

\begin{table}[ht]
\centering
\caption{Selected Metrics for Evaluation}
\label{tab:metrics}
\begin{tabularx}{\textwidth}{|l|l|X|}
\hline
\textbf{Metric} & \textbf{Unit} & \textbf{Description} \\ \hline
MTTD & Seconds & Time elapsed between fault injection (Chaos Mesh) and root cause identification. \\ \hline
Diagnostic Accuracy & Binary & Success rate in identifying the correct root cause. \\ \hline
CPU/Memory Overhead & \% & Differential resource consumption between Scenario 2 and Scenario 1. \\ \hline
\end{tabularx}
\par\medskip\ABNTEXfontereduzida\selectfont\textbf{Fonte:} Written by author (2026) \par\medskip
\end{table}

\section{Threats to Validity}
\label{sec:threats}
Consistent with experimental guidelines in software engineering, we address validity threats:
\begin{itemize}
    \item \textbf{Internal Validity:} Workload fluctuations. \textit{Mitigation:} Use of \textbf{k6} for deterministic load generation.
    \item \textbf{External Validity:} Generalization of results. \textit{Mitigation:} Use of a polyglot microservices application (Online Boutique) that mimics industry patterns (gRPC, multiple languages).
    \item \textbf{Construct Validity:} Measurement errors. \textit{Mitigation:} Automated collection of timestamps via Chaos Mesh and Prometheus to avoid human timing errors.
\end{itemize}