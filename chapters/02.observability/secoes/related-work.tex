\section{Related Work}
\label{sec:rel-work}

Regarding related works, we can highlight the work of \cite{kosinska_2023}, which seeks to chart a course for the so-called state of the art of observability for cloud-native applications through a systematic literature review during the period from 2018 to 2022. That work aims to detail various observability approaches, classifying their application by management areas (DevOps and testing), types of requirements (functional and non-functional), and ecosystem (infrastructure or microservices). Our work focuses on more pressing matters related to achieving a minimum level of observability using open-source tools in a cloud-native environment.

On the other hand, the work of \cite{usman_2022} is more closely aligned with our work, although it does not propose a solution to the aforementioned issues. Its focus is divided into two parts: the first seeks to map aspects related to the state of the art of observability in distributed environments and microservices. The second part explores which components of data telemetry can be used to achieve an ideal state of observability. However, there is a lack of concern around practical aspects, such as designing and evaluating an observability framework. This study leverages these core aspects of observability, and it reflects on different ways to achieve a viable framework.

Another line of study \cite{faseeha_2025} classifies existing solutions according to architectural patterns such as gateway, service mesh, and event-driven. Its focus emphasizes how observability research is evolving through the use of new frameworks. However, no practical approach was proposed that considered the observability pillars. Its primary concern revolves around microservice architecture, and there is confusion between what is called monitoring parameters and those pillars.

In summary, this retrospective demonstrates that observability is not merely a modern buzzword but a fundamental property rooted in control theory, which has evolved to address the stochastic nature of distributed systems. While the identified pillars—metrics, logs, traces, and events—provide the necessary raw telemetry, and advancements like eBPF offer unprecedented visibility depth, the analysis of related work reveals a critical gap: the lack of a unified, open-source reference architecture that orchestrates these components cohesively. Current literature largely addresses specific tools or architectural patterns in isolation, failing to provide a holistic blueprint for practitioners. To bridge this gap and define the environment where the proposed architecture (PANOPTES) will operate, the next chapter explores the underlying substrate of this complexity: Cloud Computing and the Kubernetes orchestration model.