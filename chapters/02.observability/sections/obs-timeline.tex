\section{Observability Timeline}
\label{sec:obs-time}

Looking at the definition of observability throughout history, it is possible to see that, from its conception in 1960 to modern times, the concern with data collection and inferring the state of applications from it has not changed; what has changed is the form and means by which this is done. However, as shown in figure \ref{fig:obs-time}, six major shifts have occurred since Kalman introduced the term, as demonstrated in table \ref{tab:obs-evo}. These changes 

\begin{figure}
    \centering
    \caption{Observability Timeline}
    \label{fig:obs-time}
    \includegraphics[width=1.0\linewidth]{images/observability/observability-timeline.png}
    \par\medskip\ABNTEXfontereduzida\selectfont\textbf{Fonte:} Created by author (2026) \par\medskip
\end{figure}

\begin{table}[ht]
\centering
\caption{Evolution of Observability - Comparison Between Historical Phases}
\label{tab:obs-evo}
\resizebox{\textwidth}{!}{%
\begin{tabular}{llllll}
\toprule
\textbf{Period} & \textbf{Paradigm} & \textbf{Architecture} & \textbf{Collected Data} & \textbf{Main Tools} & \textbf{Main Challenge} \\ \midrule
1960-1980 & Control Theory & Monolithic/Centralized & Basic Metrics (CPU, RAM) & Analog Instruments & Limited Visibility \\
1980-2010 & IT Monitoring & Client-Server & Metrics + Simple Logs & Nagios, Zabbix, Cacti & Forecasting Necessary Data \\
2010-2015 & Transition & Early Microservices & Metrics + Logs + Traces & Zipkin, Jaeger, New Relic & Service Correlation \\
2015-2018 & Observability & Distributed Cloud-Native & Logs, Metrics, Traces & Prometheus, Grafana, Loki & Standardization and Scalability \\
2018-2022 & Consolidation & Kubernetes, Serverless & Unified Telemetry & OpenTelemetry, Datadog & Vendor Integration \\
2022-2026 & Intelligence & Distributed + Edge & Telemetry + AI/ML & Cribl, Honeycomb, Datadog & Cost and Complexity \\ \bottomrule
\end{tabular}%
}
  \par\medskip\ABNTEXfontereduzida\selectfont\textbf{Fonte:} Written by author (2026) \par\medskip
\end{table}

According to \cite{kosinska_2023}, observability is often defined as consisting of three pillars: (i) metrics (collected while monitoring activities), (ii) logging, and (iii) tracing. However, as mentioned by \cite{hartikainen_2024}, the mere fact that a system possesses these three pillars does not automatically make it observable. The most value is obtained by combining these pillars and converting them into interpretable data.

Later, \cite{bhatia_2025} informs that an additional pillar is used in modern observability frameworks: events. These essential telemetry categories, when combined, facilitate comprehensive system visibility. For further clarification, the definitions of each concept will be provided in the next section.

Metrics represent the raw measurements of resource usage or behavior that can be observed and collected throughout your systems \cite{ellingwood_2017}. They can be interpreted as counters or measurements of a system characteristic during a time period; they are numeric by definition and represent aggregated data. Examples of metrics include the number of requests per second responded to by the HTTP server and the total heap memory allocated to the Java microservice.

Logging or logs are the primary data source for developers and software engineers to uncover the underlying problems in computing systems. Logs are capable of exporting information about each operation and the system state, making it possible to understand exactly what a process was doing at a given point in time \cite{beyer_2016}.

A distributed trace, more commonly known as a trace, records the paths taken by requests (made by an application or end-user) as they propagate through multi-service architectures, like microservice and serverless applications \cite{otel_primer_2025}. A trace is composed of one or more spans. A span represents a unit of work or operation. Spans track specific operations that a request performs, providing a clear picture of what happened during the time the operation was executed.

Events can be utilized to enhance a system's observability \cite{hartikainen_2024}. They are registered with specific actions, such as the execution of a function, the update of a database record, or the throwing of an exception. They are timestamped, unchangeable documentation that comes in three forms: plain text, structured, and binary.

\begin{figure}[ht]
    \centering
    \caption{Observability Pillars}
    \label{fig:obs-pillars}
    \includegraphics[width=0.75\linewidth]{images/observability/observability-pillars.png}
    \par\medskip\ABNTEXfontereduzida\selectfont\textbf{Fonte:} Created by author (2026) \par\medskip
\end{figure}

After understanding the core tenets of observability, it is necessary to make a clear distinction between monitoring and observability.

Although often used interchangeably, monitoring and observability are distinct concepts. Monitoring tracks a system's health using a predefined set of metrics and logs to detect known failures. However, this approach can overlook unforeseen issues in dynamic, distributed systems, as they do not fall into a predefined category of problems. Observability, in contrast, is the ability to investigate and determine where and why a failure occurred, especially for these unknown issues.

Therefore, monitoring is a prerequisite for observability, as the two are closely intertwined. A simple way to view it is that monitoring is the alarm warning an existing problem, while observability provides the tools to discover why the problem happened. The former alerts operators to a failure, and the latter helps identify its root cause, as shown in Figure \ref{fig:obs-vs-mon}.

\begin{figure}[ht]
    \centering
    \caption{Observability vs Monitoring}
    \label{fig:obs-vs-mon}
    \includegraphics[width=0.75\linewidth]{images/observability/monitoring-vs-observability.png}
    \par\medskip\ABNTEXfontereduzida\selectfont\textbf{Fonte:} Created by author (2026) \par\medskip
\end{figure}

There is a popular saying about software engineering that claims that if something cannot be measured, it cannot be controlled. Whenever a system needs to be monitored, an instrument or an instrumentation system must be in place to facilitate this process. For clarification, an instrument (or instrumentation system) is a device to measure some specific quantity \cite{ribbens_2013}.

In the purest sense, telemetry is a technology that allows data acquired in one location to be monitored and displayed in another \cite{dondelinger_2013}. The word telemetry is formed by combining two Greek words: \textit{tele}, meaning remote, and \textit{metron}, meaning measure.

In cloud-native environments, telemetry has become more abundant and diverse, presenting both a challenge and an opportunity to go beyond monitoring and achieve true observability, which is necessary to manage complexity. Such complexity is further enhanced through the process of instrumenting CNAs, which can be achieved using different levels of instrumentation. 

Considering the existence of multiple levels of instrumentation, it is possible to implement the instrumentation process across several components, i.e., from the application level to the platform level, where Kubernetes (k8s) is installed.

The first level is referred to as the code/application level. In this level, instrumentation is implemented directly in the application's source code. Developers use language-specific libraries or SDKs (Software Development Kits) (e.g., OpenTelemetry SDK, Micrometer, Prometheus Client Libraries) to define and expose metrics, create structured logs, and initiate/propagate tracing spans.

The second level is the platform level. This level abstracts network instrumentation away from the application. A sidecar proxy (e.g., Envoy, Linkerd) is injected into each application pod. This proxy intercepts all network traffic (incoming and outgoing) from the application container. Service Mesh (e.g., Istio) manages and configures these proxies to collect telemetry in a standardized way.

The third level is the host/node level. A software agent runs as a DaemonSet (one instance per cluster node). This agent collects metrics and logs from the node's operating system and running containers, typically reading information from virtual file systems such as Prometheus Node Exporter and cAdvisor.

The final level is the kernel level. The extended Berkeley Packet Filter (eBPF) is a Linux kernel technology that enables programs to run safely and efficiently in a sandbox within the kernel itself. Without altering application or kernel code, these programs can hook into low-level instrumentation points, such as system calls, network functions, and process scheduler events. Therefore, it can be classified as an instrumentation at the kernel level in Linux \cite{rise_2023}.

In the next section, we will dive deeper into recent work on observability in cloud environments.