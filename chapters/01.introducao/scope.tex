\section{OUT OF SCOPE}
\label{sec:scope}

To ensure the feasibility of this research and maintain a coherent focus on the architectural aspects of observability in Kubernetes, the following topics are explicitly excluded from the scope of this thesis:

\begin{itemize}
    \item \textbf{Development of New Monitoring Tools:} This research does not aim to develop new monitoring agents, time-series databases, or log collectors from scratch. The focus is on the \textit{architectural integration} and orchestration of existing, mature open-source projects (e.g., OpenTelemetry, Prometheus, Jaeger).
    
    \item \textbf{Comparison with Proprietary Solutions:} While commercial APM (Application Performance Monitoring) solutions (e.g., Datadog, Dynatrace, New Relic) are prevalent in the industry, this study restricts its comparative analysis to the Kubernetes \textit{baseline} (vanilla installation). Comparing open-source architectures against proprietary SaaS platforms would introduce bias due to feature disparity and licensing costs.
    
    \item \textbf{Auto-Remediation and AIOps:} Although observability is a prerequisite for automated healing (AIOps), this thesis focuses solely on the \textit{detection and diagnosis} phases of the incident lifecycle. Automated remediation strategies (e.g., restarting pods based on custom metrics) are considered future work.
    
    \item \textbf{Non-Containerized Environments:} The proposed reference architecture is designed specifically for cloud-native applications running on Kubernetes. Legacy systems, monolithic applications on virtual machines, or serverless functions (FaaS) are outside the scope of this work.
\end{itemize}