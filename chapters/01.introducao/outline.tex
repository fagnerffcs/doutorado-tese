\section{OUTLINE}
\label{sec:outline}

To present these contributions, the remainder of this thesis is structured as follows: Chapter \ref{chap:cloud-computing} provides the theoretical background on Cloud-Native computing, the Kubernetes orchestration model, and the three pillars of observability (metrics, logs, and distributed tracing). It also discusses the concepts of Chaos Engineering used to validate system resilience and diagnosability. Chapter \ref{chap:observability} reviews related work and the state of the art in observability architectures, analyzing existing gaps in default Kubernetes installations and comparing different architectural approaches found in the literature. Chapter \ref{chap:methodology} details the research methodology, describing the experimental design, the definition of the baseline and experimental groups, the fault injection scenarios (Chaos Engineering), and the metrics selected to evaluate the effectiveness of the proposed solution. Chapter 5 introduces the proposed Reference Architecture for Kubernetes Observability. It details the architectural decisions, the selection of open-source components (such as OpenTelemetry, Prometheus, and Jaeger), and the strategy for semantic integration of telemetry data. Chapter 6 presents the results of the controlled experiments. It provides a comparative analysis of the baseline and the proposed architecture, discussing the impact on fault-detection time and diagnostic precision based on the collected data. Chapter 7 concludes the thesis with a summary of the findings, a discussion on the limitations of the study, and directions for future work in the field of cloud-native observability.