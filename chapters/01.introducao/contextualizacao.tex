\section{Contextualization}
\label{sec:context}

Observability is the capability to continuously generate and discover actionable insights from signals in the system under observation, with the goal of influencing the system \cite{hausenblas_2023}. It can be interpreted as a feedback loop in which the system generates data, which agents collect from sources and transmit to destinations for analysis. Based on this analysis, it will prompt an action (either autonomous or user-initiated) to reset the cycle.

The cloud environment is no longer a mere promise, but it is a reality faced by many enterprises around the world. AWS, for example, has been the leader in the Gartner Magic Quadrant for Strategic Cloud Platform Services for 15 years \cite{aws_cloud_computing_2025}. This scenario reinforces the concept that cloud adoption is a cornerstone for digital transformation \cite{kao_2024}, especially for small and medium enterprises (SMEs) \cite{ahmad_2025}.

Many advantages are presented when implementing this architecture: cost optimization \cite{plotnikovs_kodors_2017}, scalability, elasticity, and efficiency \cite{lehrig_2015}, optimized configuration management tools \cite{hintsch_2016}, and automatic software integration \cite{apostu_2013}. The scenario looks promising, as these capabilities are presented without significant complexity.

However, there is a hidden element of complexity behind all this near-perfect world: how to discover what lies beneath an application failure when all infrastructure indicators or signals (telemetry data) point to no failure at all?
 