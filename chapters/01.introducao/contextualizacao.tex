\section{Contextualization}
\label{sec:context}

Observability according to \cite{hausenblas_2023} is the capability to continuously generate and discover actionable insights based on signals from the system under observation, with the goal of influencing the system. It can be interpreted as a feedback loop where system generate data, this data is collected by agents from sources and transmitted to destinations where it will be analyzed. Depending of this analyze, it will prompt an action (autonomous or by user) which can reset the cycle.

The cloud environment is no longer a mere promise, but it is a reality faced by many enterprises around the world. AWS, for example, has been the leader of Gartner Magic Quadrant for Strategic Cloud Platform Service for 15 years \cite{aws_cloud_computing_2025}. Many advantages are presented when implementing this architecture: scalability seemly infinity, elasticity at hand, minimal configuration and unified management. The scenario looks promising since these several capabilities are presented without a great complexity along.

However, there is a hidden element of complexity behind all this near perfect world: how to discover what lies beneath an application failure when all infrastructure indicators or signals point to no failure at all?
 