\section{PROBLEM STATEMENT}
\label{sec:problem}

While several tools emerged to address different pillars of observability, such as in microservices \cite{faseeha_2025}, the Kubernetes (K8S) architecture and default installation \cite{johansson_2022} do not include such tools. In fact, there is a long list of ways to install a K8S cluster \cite{luksa_2018}. Therefore, to address this predicament, it is necessary to use an automated, secure tool to install observability tools, such as Helm Charts.

Considering this scenario, this study aims to investigate how to tackle this complexity using open-source tools to monitor cloud-native apps. This research delves into one central question and three auxiliary questions surrounding observability architecture as explained in section \ref{subsec:research-questions}.

\subsection{Research Questions}
\label{subsec:research-questions}

\textbf{Main Research Question} \textit{How can a reference architecture for observability in Kubernetes, based on open standards, enhance the ability to diagnose faults in microservices compared to the platform's native monitoring mechanisms?}

To provide a comprehensive answer to this overarching question, the following secondary research questions have been delineated:

\textbf{Secondary Research Question 1} \textit{What are the minimum architectural requirements and open-source components needed to compose a complete observability layer (Metrics, Logs, and Tracing) in Kubernetes orchestrators?}

\textbf{Secondary Research Question 2} \textit{How can we instantiate a reference architecture using open-source tools that ensures semantic integration between the pillars of observability without excessively burdening cluster resources?}

\textbf{Secondary Research Question 3} \textit{What is the impact of the proposed architecture on the effectiveness of fault detection and diagnosis compared to a standard (Baseline) Kubernetes installation?}

\subsection{Research Objectives}

To answer the research questions, the main objective of this thesis is to \textbf{propose and validate} an observability reference architecture for cloud-native applications on Kubernetes.

Specifically, the specific objectives are:

\textbf{SO1} \textit{To \textbf{identify and define} the essential architectural requirements and components necessary to establish a cohesive observability layer.}

\textbf{SO2} \textit{To \textbf{design and implement} an open-source reference architecture instance that integrates metrics, logs, and traces with minimal resource overhead.}

\textbf{SO3} \textit{To \textbf{evaluate} the proposed architecture's effectiveness in diagnosing faults through controlled experiments (Chaos Engineering), comparing it against a Kubernetes baseline.}