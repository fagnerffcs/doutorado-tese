\chapter{INTRODUCTION}
\label{chap:intro}

The evolution of software engineering has shifted from the construction of monolithic, deterministic applications to the orchestration of highly distributed, ephemeral, and dynamic systems. In this contemporary landscape, characterized by the adoption of Cloud-Native paradigms, the complexity of software architecture has transcended the boundaries of code organization to encompass the intricate interactions between microservices, container orchestrators, and infrastructure as code. As systems grow in scale and heterogeneity, the challenge shifts from merely designing correct software to ensuring its operability and reliability under uncertainty \cite{newman_2015, brewer_2000}.

While traditional monitoring focused on the collection of predefined metrics to detect known failures, the stochastic nature of modern Kubernetes-based environments demands a more profound property: Observability. Often conflated with telemetry collection, true observability is an intrinsic property of a system that allows its internal state to be inferred solely from its external outputs. This concept is not a novel invention of the cloud era but has deep roots in the General Theory of Control Systems.

In his seminal work, \cite{kalman_1959} formalized the mathematical dualism between control and observation. He posited that for a system (``the plant'') to be effectively controlled—or in modern terms, orchestrated and self-healed—it must be ``completely observable''. Kalman defined that a state is observable if its exact value can be determined from measurements of the output signal over a finite interval. This theoretical foundation challenges the current ad-hoc approach to observability in the software industry, where the focus is often on the volume of data collected (logs, metrics, traces) rather than the mathematical completeness of the information required to understand system behavior.

However, translating these control theory principles into the domain of Cloud-Native applications presents significant architectural hurdles. The ``plant'' is no longer a linear, stationary system with a single input and output, as simplified in early control models, but a chaotic mesh of services with varying lifecycles. The decision-making process for designing an observability architecture in this context is fraught with ``bounded rationality'' \cite{simon_1990}, where architects must navigate a confusing landscape of open-source tools—such as Prometheus, OpenTelemetry, and Jaeger—without a unified reference model.

Therefore, just as architectural decisions in software design dictate the system's lifespan and maintainability \cite{Tang_Razavian_Paech_Hesse_2017}, the architectural decisions regarding observability determine the system's ``controllability''. Without a structured approach to implement observability that respects the theoretical imperatives of control systems, organizations risk constructing opaque systems that resist diagnosis and automated recovery. This thesis aims to bridge this gap, proposing a reference architecture that aligns the rigor of Kalman's control theory with the practicalities of the open-source Kubernetes ecosystem. 

\input{chapters/01.introduction/motivation}

\input{chapters/01.introduction/contextualization}
  
\section{PROBLEM STATEMENT}
\label{sec:problem}

While several tools emerged to address different pillars of observability, such as in microservices \cite{faseeha_2025}, the Kubernetes (K8S) architecture and default installation \cite{johansson_2022} do not include such tools. In fact, there is a long list of ways to install a K8S cluster \cite{luksa_2018}. Therefore, to address this predicament, it is necessary to use an automated, secure tool to install observability tools, such as Helm Charts.

Considering this scenario, this study aims to investigate how to tackle this complexity using open-source tools to monitor cloud-native apps. This research delves into one central question and three auxiliary questions surrounding observability architecture as explained in section \ref{subsec:research-questions}.

\subsection{Research Questions}
\label{subsec:research-questions}

\textbf{Main Research Question} \textit{How can a reference architecture for observability in Kubernetes, based on open standards, enhance the ability to diagnose faults in microservices compared to the platform's native monitoring mechanisms?}

To provide a comprehensive answer to this overarching question, the following secondary research questions have been delineated:

\textbf{Secondary Research Question 1} \textit{What are the minimum architectural requirements and open-source components needed to compose a complete observability layer (Metrics, Logs, and Tracing) in Kubernetes orchestrators?}

\textbf{Secondary Research Question 2} \textit{How can we instantiate a reference architecture using open-source tools that ensures semantic integration between the pillars of observability without excessively burdening cluster resources?}

\textbf{Secondary Research Question 3} \textit{What is the impact of the proposed architecture on the effectiveness of fault detection and diagnosis compared to a standard (Baseline) Kubernetes installation?}

\subsection{Research Objectives}

To answer the research questions, the main objective of this thesis is to \textbf{propose and validate} an observability reference architecture for cloud-native applications on Kubernetes.

Specifically, the specific objectives are:

\textbf{SO1} \textit{To \textbf{identify and define} the essential architectural requirements and components necessary to establish a cohesive observability layer.}

\textbf{SO2} \textit{To \textbf{design and implement} an open-source reference architecture instance that integrates metrics, logs, and traces with minimal resource overhead.}

\textbf{SO3} \textit{To \textbf{evaluate} the proposed architecture's effectiveness in diagnosing faults through controlled experiments (Chaos Engineering), comparing it against a Kubernetes baseline.}
 
\section{Out of Scope}
\label{sec:scope}
Texto texto texto texto texto texto texto texto texto texto texto texto texto texto texto texto texto texto texto texto texto texto texto texto texto texto texto texto texto texto texto texto texto texto texto texto.

Texto texto texto texto texto texto texto texto texto texto texto texto texto texto texto texto texto texto texto texto texto texto texto texto texto texto texto texto texto texto texto texto texto texto texto texto.



\section{STATEMENT OF CONTRIBUTION}
\label{sec:statement}

\section{OUTLINE}
\label{sec:outline}
