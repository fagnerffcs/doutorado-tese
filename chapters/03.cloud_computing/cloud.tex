\chapter{CLOUD COMPUTING}
\label{chap:cloud-computing}

The transition from monolithic systems to cloud-native paradigms \cite{gannon_2017} serves as the primary catalyst for the observability challenges discussed in the previous chapter. Ideally, Cloud Computing is not just a deployment target but a dynamic operational model characterized by elasticity, on-demand self-service, and resource pooling \cite{mell_grance_2011}. However, these characteristics introduce distinct layers of abstraction and transience that obscure system behavior \cite{jamshidi_2018}. To properly contextualize the environment in which the PANOPTES architecture operates, this chapter establishes the theoretical foundations of Cloud-Native computing. It examines the pivotal role of Kubernetes as the \textit{de facto} operating system for the cloud \cite{luksa_2018, burns_2016} and analyzes Infrastructure as Code (IaC) as the mechanism for managing this complexity \cite{morris_2016}. Understanding these components is a prerequisite to addressing the research questions regarding fault diagnosis and architectural integration.

\subsection{Service Models}
\label{subsec:service-models}

Cloud computing is characterized by a layered service model that dictates the level of abstraction and control granted to the user. According to the foundational definition by the National Institute of Standards and Technology (NIST) \cite{mell_grance_2011}, these models are categorized into three distinct layers: Software as a Service (SaaS), Platform as a Service (PaaS), and Infrastructure as a Service (IaaS). This taxonomy is often visualized as a stack, where each layer inherits capabilities from the one below it, presenting a trade-off between extensibility and ease of management \cite{armbrust_2010}.

\begin{itemize}
    \item \textbf{Software as a Service (SaaS):} In this model, the consumer uses the provider's applications running on a cloud infrastructure. The underlying infrastructure, including network, servers, operating systems, and application capabilities, is entirely managed by the vendor. This model offers the highest level of abstraction but the least control \cite{zhang_2010}.
    
    \item \textbf{Platform as a Service (PaaS):} This model provides a deployment environment for developers to create and host applications using languages, libraries, and tools supported by the provider. The consumer manages the deployed applications and possibly configuration settings for the application-hosting environment, but does not control the underlying infrastructure (servers, OS, storage) \cite{mell_grance_2011}.
    
    \item \textbf{Infrastructure as a Service (IaaS):} This model provisions fundamental computing resources such as processing, storage, and networks, where the consumer can deploy and run arbitrary software, including operating systems and applications. The consumer does not manage the underlying cloud infrastructure but controls operating systems, storage, and deployed applications \cite{mell_grance_2011}. As noted by \citeonline{armbrust_2010}, IaaS most closely mimics traditional datacenter hardware, providing the raw flexibility required for complex system engineering.
\end{itemize}

\subsubsection{Adopted Model for this Research}

For the specific purpose of this thesis—instantiating and validating the \textbf{PANOPTES} observability architecture—we adopt the \textbf{Infrastructure as a Service (IaaS)} model. This decision is driven by the requirement for granular access to the Kubernetes control plane, Container Network Interface (CNI), and kernel-level instrumentation points (eBPF), which are often abstracted away in PaaS offerings (Managed Kubernetes).

To ensure reproducibility and simulate a private IaaS environment for the baseline experiments, we employ a local virtualization approach orchestrated by \textbf{Vagrant} and configured via \textbf{Ansible}. This setup enables the provisioning of virtual machines that closely mimic bare-metal servers, providing a consistent testbed for comparing cloud-native behaviors with local deployments.

\subsection{Cloud-Native}
\label{subsec:cna}
The definition of the cloud-native term is usually related to something built in the cloud, rather than on-premises \cite{gannon_2017}. However, this definition is incomplete, as cloud-native refers to the software method of creating, deploying, and managing modern applications in cloud computing environments \cite{aws_2025}. Therefore, it is possible to host cloud-native applications (CNAs) both in the cloud and on-premises, where both styles are sometimes used simultaneously.

The growth of microservices architecture, in conjunction with this model of hosting applications, has introduced a new level of complexity in analyzing problems within the underlying infrastructure. Where organizations once relied on centralized, monolithic systems, modern environments feature distributed frameworks characterized by containerized workloads, independently deployable microservices, and multi-cloud infrastructure arrangements \cite{bhatia_2025}.

\section{Kubernetes}
\label{sec:k8s}

The transition from monolithic applications to cloud-native microservices architectures, orchestrated by Kubernetes (K8s), has introduced significant operational complexity. Unlike traditional environments, where monitoring relied on checking the binary (up/down) states of static servers, Kubernetes manages ephemeral, dynamic containers, making failures unpredictable and challenging to diagnose with only conventional metrics \cite{usman_2022}.

In this scenario, observability transcends traditional monitoring. It is defined as the property of a system that allows understanding its internal state from its external outputs (telemetry), enabling the correlation of signals to identify "unknown unknowns" \cite{hausenblas_2023,creane_2022}. For applications orchestrated via Kubernetes, a reference architecture should address the collection, processing, and correlation of data across multiple layers: infrastructure (nodes), orchestrator (K8s), and application (microservices).

\subsection{Layers of the Observability Architecture}

The proposed architecture is based on a telemetry data pipeline composed of four fundamental stages: Generation (Instrumentation), Collection, Processing/Storage, and Analysis \cite{kosinska_2023,young_2024}.

\subsubsection{Instrumentation and Signal Generation Layer}
The basis of observability lies in proper instrumentation. In the context of Kubernetes, it is possible to extract data from three primary sources:

\textbf{Infrastructure and Nodes (Node-level):} This involves capturing CPU, memory, disk, and network metrics from the underlying operating system. The Kubelet, the agent that runs on each node, natively includes \textit{cAdvisor}, which collects resource usage metrics from containers and exposes them for consumption \cite{burns_2019,hausenblas_2023}.

\textbf{Kubernetes Control Plane:} Components such as the API Server, Controller Manager, and Scheduler expose vital metrics and audit logs that record the sequence of activities and decisions made by the cluster, essential for security and troubleshooting \cite{creane_2022,muschko_2020}.

\textbf{Application (Microservices):} Business metrics, structured logs, and distributed traces are considered golden signals \cite{turnbull_2018}; therefore, applications should be instrumented to emit these signals. OpenTelemetry (OTel) has become the industry standard for this instrumentation, unifying signal generation without vendor dependency (vendor-agnostic) \cite{young_2024,flanders_2024}.

\subsubsection{Collection Patterns in Kubernetes}

Data collection in a Kubernetes cluster can be implemented through different architectural patterns, each with specific advantages:

\textbf{DaemonSet (Agent per Node):} This is the most efficient pattern for collecting infrastructure logs and metrics. An agent (such as Fluentd or OpenTelemetry Collector) is deployed on each node of the cluster. It reads logs directly from the `/var/log/containers` directory (or `stdout`/`stderr` streams from Docker/containerd) and collects *host* metrics, enriching the data with Kubernetes metadata (such as *namespace*, *pod name*, *labels*) \cite{wilkins_2019, creane_2022, dubey_2021}.

\textbf{Sidecar:} A sidecar container is deployed in the same Pod as the application. Although it consumes more resources, it helps convert logs from legacy applications or for *service mesh* proxies that collect detailed network metrics \cite{burns_2019,ibryam_2019}.

\textbf{Operators (Kubernetes Operators):} The Operator pattern is widely used to manage the observability lifecycle. For example, the Prometheus Operator \cite{operator_hub_2026} or the OpenTelemetry Operator \cite{otel_operator_2026} automate configuration injection, agent management, and service discovery for metric scraping \cite{dobies_2020,flanders_2024}.

\subsection{The Pillars of Observability in Kubernetes}

Previously, it was presented that metrics, traces, and logs are defined as pillars of observability. In K8s, these constructs are inconsistently defined and scattered throughout its architecture for other purposes, such as the Metrics Server, which is designed to enable autoscaling. However, before elaborating on finer details of K8s, it is necessary to know its current state. 

\subsubsection{Metrics and the Prometheus Ecosystem} 
For numerical time-series data, Prometheus is the \textit{de facto} standard in the Cloud Native Computing Foundation (CNCF) ecosystem \cite{yeruva_2021}. The reference architecture uses the pull model, where the Prometheus server performs the scraping of metrics exposed by the `/metrics` endpoints of the services and Kubernetes itself (via kube-state-metrics) \cite{dubey_2021}.

For horizontal pod scalability (HPA - Horizontal Pod Autoscaler), the Metrics Server is an essential component that aggregates resource usage metrics and makes them available via the Kubernetes metrics API \cite{muschko_2020}.

\subsubsection{Logs and Centralized Aggregation}
Logs in distributed environments cannot be accessed in isolation via 'kubectl logs` in production. The architecture requires a log forwarder (such as Fluentd or Fluent Bit) that centralizes the output streams. These logs must be processed to extract structured fields correlated with Kubernetes metadata before being sent to a storage backend (such as Elasticsearch, Loki, or OpenSearch) \cite{wilkins_2019,ibryam_2019}.

\subsubsection{Distributed Tracing}
To understand latency and request flow between microservices, distributed tracing is mandatory. Tools like **Jaeger** or **Tempo** receive spans generated by applications instrumented via OpenTelemetry. The OpenTelemetry *Collector* component plays a crucial role here, receiving traces in protocols such as OTLP, processing them (e.g., sampling, filtering sensitive data), and exporting them to the visualization backend \cite{hausenblas_2023,young_2024}.

\subsubsection{The Role of OpenTelemetry (OTel) in the Architecture}
OpenTelemetry (OTel) acts as the unifying layer in the reference architecture. It solves the problem of tool fragmentation by providing a single standard for APIs, SDKs, and protocols (OTLP) \cite{flanders_2024}.

A central component is the OpenTelemetry Collector, which can be deployed in two main modes in Kubernetes: Agent Mode (DaemonSet) or Gateway Mode (Deployment). In the first case, it runs on each node to collect logs and metrics from the host, minimizing network latency. In the second one, it is a centralized cluster of collectors that receives data from the agents, performs heavy processing (such as tail-based sampling and aggregation), and exports to the back.

\section{Infrastructure as Code}
\label{sec:iac}