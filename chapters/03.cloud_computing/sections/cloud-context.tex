\section{Cloud Context in this Work}
\label{sec:cloud-context}

The implementation of a robust observability architecture is intrinsically linked to the underlying computing paradigm. Before addressing specific orchestration tools like Kubernetes, it is necessary to establish the architectural boundaries and operational constraints within which this research operates.

Cloud computing is not a monolithic concept; it presents a spectrum of abstractions that trade control for convenience. In this section, we delineate the fundamental Service Models defined by the literature, explicitly justifying the adoption of the Infrastructure as a Service (IaaS) model for this thesis. This choice is critical as it grants the necessary access to low-level components (such as the kernel and network interfaces) required for deep instrumentation. Furthermore, exploring the Cloud-Native paradigm involves analyzing how its principles of immutability and distribution introduce the very complexity that necessitates the observability solutions proposed in later chapters.

\subsection{Service Models}
\label{subsec:service-models}

Cloud computing is characterized by a layered service model that dictates the level of abstraction and control granted to the user. According to the foundational definition by the National Institute of Standards and Technology (NIST) \cite{mell_grance_2011}, these models are categorized into three distinct layers: Software as a Service (SaaS), Platform as a Service (PaaS), and Infrastructure as a Service (IaaS). This taxonomy is often visualized as a stack, where each layer inherits capabilities from the one below it, presenting a trade-off between extensibility and ease of management \cite{armbrust_2010}.

\begin{itemize}
    \item \textbf{Software as a Service (SaaS):} In this model, the consumer uses the provider's applications running on a cloud infrastructure. The underlying infrastructure, including network, servers, operating systems, and application capabilities, is entirely managed by the vendor. This model offers the highest level of abstraction but the least control \cite{zhang_2010}.
    
    \item \textbf{Platform as a Service (PaaS):} This model provides a deployment environment for developers to create and host applications using languages, libraries, and tools supported by the provider. The consumer manages the deployed applications and possibly configuration settings for the application-hosting environment, but does not control the underlying infrastructure (servers, OS, storage) \cite{mell_grance_2011}.
    
    \item \textbf{Infrastructure as a Service (IaaS):} This model provisions fundamental computing resources such as processing, storage, and networks, where the consumer can deploy and run arbitrary software, including operating systems and applications. The consumer does not manage the underlying cloud infrastructure but controls operating systems, storage, and deployed applications \cite{mell_grance_2011}. As noted by \citeonline{armbrust_2010}, IaaS most closely mimics traditional datacenter hardware, providing the raw flexibility required for complex system engineering.
\end{itemize}

\subsubsection{Adopted Model for this Research}
\label{subsubsec:iaas}
For the specific purpose of this thesis—instantiating and validating the PANOPTES observability architecture—we adopt the Infrastructure as a Service (IaaS) model. This decision is driven by the requirement for granular access to the Kubernetes control plane, Container Network Interface (CNI), and kernel-level instrumentation points (eBPF), which are often abstracted away in PaaS offerings (Managed Kubernetes).

To ensure reproducibility and simulate a private IaaS environment for the baseline experiments, we employ a local virtualization approach orchestrated by \textbf{Vagrant} and configured via \textbf{Ansible}. This setup enables the provisioning of virtual machines that closely mimic bare-metal servers, providing a consistent testbed for comparing cloud-native behaviors with local deployments.

\subsection{Cloud-Native}
\label{subsec:cna}
The definition of the cloud-native term is usually related to something built in the cloud, rather than on-premises \cite{gannon_2017}. However, this definition is incomplete, as cloud-native refers to the software method of creating, deploying, and managing modern applications in cloud computing environments \cite{aws_2025}. Therefore, it is possible to host cloud-native applications (CNAs) both in the cloud and on-premises, where both styles are sometimes used simultaneously. According to \cite{davis_2019}, Cloud-native software is highly distributed, must operate in a constantly changing environment, and is itself constantly changing.

The growth of microservices architecture, in conjunction with this model of hosting applications, has introduced a new level of complexity in analyzing problems within the underlying infrastructure. Where organizations once relied on centralized, monolithic systems, modern environments feature distributed frameworks characterized by containerized workloads, independently deployable microservices, and multi-cloud infrastructure arrangements \cite{bhatia_2025}.

\subsubsection{Pillars of Cloud Native Apps}

According to \citeonline{microsoft_cna_2026}, there are five fundamental pillars of cloud infrastructure as depicted in figure \ref{fig:cna-pillars}. Modern design is related to Twelve Factor \cite{twelve_factor_2017}. They are a set of principles and practices that developers must follow to build applications for modern cloud environments. A microservice architecture is composed of a set of small services, each running in its own process and communicating with lightweight mechanisms \cite{di_francesco_2019}. Containers are ``...a computing context that uses functionality
from a host that it's running on.'' \cite{davis_2019}. Backing services are defined as many different auxiliary resources, such as data storage, message brokers, monitoring, and identity services \citeonline{microsoft_cna_2026}. Finally, the concept of automation is directly related to the IaC section and, as will be explored in more depth in that topic, can be defined as the ability to execute tasks in a deterministic, sequential, and parameterizable manner, ensuring consistency across environments \cite{morris_2016}.

\begin{figure}[ht]
    \centering
    \includegraphics[width=1.0\linewidth]{images/cloud-computing/cloud-native-foundational-pillars.png}
    \caption{CNA Pillars}
    \label{fig:cna-pillars}
\end{figure}